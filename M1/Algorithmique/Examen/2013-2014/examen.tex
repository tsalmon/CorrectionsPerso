% Pour generer et afficher le pdf en ligne de commande : pdflatex examen.tex && evince evince.pdf
% Ou utiliser texmaker

\documentclass[a4paper]{article}
\usepackage[utf8]{inputenc}
\usepackage[a4paper]{geometry}
\usepackage{algpseudocode}
\usepackage{amsfonts}
\usepackage{amsmath}
\usepackage{mathtools}
\usepackage[francais]{babel}

\title{Correction de l'examen d'Algorithmique}
\author{Thomas Salmon}
\date{10/03/2014}

\begin{document}
\maketitle
\section{Exercice 1 Diviser pour régner}
\subsection{Algorithme}
\begin{algorithmic}
\Function{Ex1}{tab: tableau d'entiers}
\If {$taille(tab) = 2$}
	\If {$tab[0] > tab[1]$}
	
		\State \Return $[tab[1], tab[0]]$
	\Else
	
		\State \Return $[tab[0], tab[1]]$
	\EndIf
\ElsIf {$taille(tab) = 1$}
	\State \Return $[tab[0]]$
\ElsIf {$taille(tab) = 0$}
	\State \Return $[]$
\Else
\State $t1\gets Ex1(tab[0, \dots, taille(tab)/2]$
\State $t2\gets Ex1(tab[taille(tab)/2, \dots, taille(tab)]$
\State $min_v \gets  min(t1+t2)$
\State	$max_v \gets max(t1+t2)$
\State \Return $[min_v, max_v]$
\EndIf
\EndFunction
\end{algorithmic}
\subsection{Complexité}
On rappel l'équation du Master Theorem:

$T(n) = aT(\frac{n}{b}) + f(n)$

Dans cet algorithme, on coupe a chaque fois - qu'il contient plus de deux élèments - 

le tableau en deux sous-parties ($a = 2$).

Chacune de ces parties a donc pour taille la moitié du tableau ($b = 2$).

les méthodes $min$ et $max$ sont de complexité $O(n)$, ce sont les seules itérations que l'on fait 

dans notre algo (en dehors des recursions) ($f(n) = O(n)$).\\

On a donc l'équation: 

$T(n) = 2T(\frac{n}{2}) + O(n)$\\

On rappel l'égalité

$log_2(X) = \frac{ln(X)}{ln(2)}$\\

Donc $log_2(2) = 1$

$f(n) = O(n^c), c = 1$

Ainsi $log_2(2) = c$, donc nous sommes dans le cas 2

$T(n) = \Theta(n^c log^k(n))$

$T(n) = \Theta(n.log(n))$
\subsection{Justification}

Ce programme renvoit  un coupe de valeur si il y a au moins 2 élèments dans le tableaux, 

cependant il faut l'appeller avec une autre fonction pour renvoyer une erreur si il y en a 

moins ou pas.
\section{Ex2}
\section{Exercice 3: Calcul des notes}
On commence par trier les notes par ordre décroissant

On incrémente un compte de crédit qui augment tant qu'il n'a pas atteint 30

Quand il vient de dépasser 30, on retranche a la derniere matiere comptabilité, les crédits en trop.
\section{Exercice 4}

\end{document}
