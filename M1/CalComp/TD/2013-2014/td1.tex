%pour generer le pdf en ligne de commande : pdflatex td1.tex && evince td1.pdf
\documentclass[a4paper]{article}

\usepackage[utf8]{inputenc}
\usepackage[a4paper]{geometry}
\usepackage{amsfonts}
\usepackage{amsmath}
\usepackage{mathtools}
\usepackage[francais]{babel}

\title{Correction du TD1 de Calculabilité et Complexité}
\author{Thomas Salmon}
\date{03/02/2014}

\begin{document}
\maketitle

\begin{math}
  \begin{array}{ll}
    \mathbb{E} $ dénombrable$
    &\equiv \exists f $ bijection $ f:\mathbb{E} \rightarrow \mathbb{N}\\
    &\equiv \exists g $ bijection $ g:\mathbb{N} \rightarrow \mathbb{E}\\
    &\equiv \exists h $ bijection $ h:\mathbb{E} \rightarrow \mathbb{F} $ avec $ \mathbb{F} $ denombrable $ \\
    \varphi $ une bijection$\\
  \end{array} 
\end{math}

\section{}
\begin{math}
  \mathbb{P} = \{0, 2, \dots \} $ est dénombrable$ \\
  f:\mathbb{N}\rightarrow \mathbb{P}\\
  f(n)=2n \\
  f $ injective si $ f(n_1) = f(n_2) $ alors $ 2n_1 = 2n_2 \Rightarrow n_1 = n_2\\
  $si $ \forall p \in \mathbb{P} , \exists n \in \mathbb{N} $ tel que $ f(n) = p $ il suffit de prendre $ p/2 
\end{math}

\section{}
\begin{math}
  \mathbb{P} = \{2, 3, 5, 7, 11, 13, \dots \} $ est dénombrable$ \\
  f:\mathbb{N}\rightarrow \mathbb{P} \\
  f(n) = n^{eme} $ nombre premier $  \\ 
  f(n) =
  \left\{
  \begin{array}{rcr}
    2 \\
    $NEXT PRIME$(f(n-1))\\
  \end{array}
  \right.\\
  $NEXT PRIME$(x) = $le plus petit premier  $ > x
\end{math}

\section{}
\begin{math}
  f(n) = 
  \left\{
  \begin{array}{rcr}
    num(A) \\
    num\{a \in A | a > f(n-1)\}\\
  \end{array}
  \right.\\
\end{math}
\newpage
\section{}
\begin{math}
  E_1, E_2 $ dénombrable, montrez que $ E_1 \cup E_2 $ denombrable $ \\
  E_1 = \{e_0^{1}, e_1^{1}, e_2^{1}, \dots \} \\
  E_2 = \{e_0^{2}, e_1^{2}, e_2^{2}, \dots \} \\
  \\
  E_1 \cup E_2 = \{e_0^{1}, e_0^{2}, e_1^{1}, e_1^{2}, \dots \}\\
  $Soit $ e \in E_1  \cup E_2 $, pour calculer $ f:E_1 \cup E_2 \rightarrow \mathbb{N} \\
  $soit $ e $ est dans $ E_1 $ soit dans $ E_2 $, si $ e $ est dans $ E_1 $, il est dans $ \{e_0^{1}, e_1^{1}, e_2^{1}, \dots \} $,$\\
  $alors $ \exists j $ tel que $ e = e_j^{1} $ donc$\\
  f(e) = 
  \left\{
  \begin{array}{rcr}
    2j $ si $ e \in E_1  \\
    2j+1 $ si $ e \in E_2  \\
  \end{array}
  \right.
\end{math}
\section{}
\begin{math}
  \begin{array}{ll}
    E_1 = e_0^{1}, e_1^{1}, \dots \\
    E_2 = e_0^{2}, e_1^{2}, \dots \\
    E_3 = e_0^{3}, e_1^{3}, \dots \\
    \dots \\
    E_m = e_0^{m}, e_1^{m}, \dots \\
  \end{array} \\
  $Soit $ e \in E_1 \cup E_2 \cup \dots \cup E_m $ la définition de $ f(e) $ est$ \\
  f(e) =   
  \left\{
  \begin{array}{rcr}
    mj $ si $ E_1\\
    \dots \\
    mj+m(-1) $ si $ E_m\\
  \end{array}
  \right.\\
\end{math}

\section{}
idem

\section{}
\begin{math}
f(\mathbb{N}) = \mathbb{Z} $, $ n \in \mathbb{N}
f(n) = 
\left\{
\begin{array}{rcr}
  n/2 $ si $ n $ est pair$ \\
  (n+1)/2 $ sinon$
\end{array}
\right.\\
\end{math}
\end{document}
