%pour generer et afficher le pdf en ligne de commande : pdflatex td1.tex && evince td1.pdf
\documentclass[a4paper]{article}

\usepackage[utf8]{inputenc}
\usepackage[a4paper]{geometry}
\usepackage[]{amssymb}
\usepackage{amsfonts}
\usepackage{amsmath}
\usepackage{mathtools}
\usepackage[francais]{babel}

\title{Correction du TD5 d'Algorithmique Avancée}
\author{Thomas Salmon}
\date{06/02/2014}

\begin{document}
\maketitle
\section{Exercice 1}
\begin{description}
    \item[Acte juridique] 
		\begin{description}
    		\item[]
		    \item[Fait juridique] Un événement qui crée, transmet ou éteint un droit sans qu'une personne n'ait voulu ce résultat.
		\end{description}
    \item[Fait juridique] Un événement qui crée, transmet ou éteint un droit sans qu'une personne n'ait voulu ce résultat.
\end{description}

\begin{tabular}{ll}
Montrons que \textbf{(axiome strict} " & $\Leftrightarrow$ \textbf{axiome large}\\
												              &  $\Leftarrow$ trivial\\
\end{tabular}


\begin{description}
    \item[Montrons $\Rightarrow$] 
		\begin{description}
    		\item[]
		    \item[] soit I et J tels que $|J| > |I|$
			\item[] soit $J' \in J$ et $|J'| = 1 + |J|$
		\end{description}
    \item[] On applique l'\textbf{axiome strict}
    \item[] 
    	\begin{description}
			\item[] $\exists x \in J' - I$  tel que $I + x \in \mathcal{I}$ 	
			\begin{description}
				\item[]
				\item[] $\Rightarrow x \in J - I CQFD$
			\end{description}
    	\end{description}
\end{description}

\section{Exercice 2}
Supposons qu'il existe 2 bases $B$ et $B'$ avec $|B| > |B'|$

\textbf{axiome large} $\Rightarrow \exists x \in B - B' $ tel que $B' + x$ est indépendant mais

$B' \nsubseteq  B' + x $ ce qui contredit la maximalité de $B$ 
\section{Exercice 3}
Matrice sur $\{0, 1\}$

\begin{tabular}{| l | l | l | l |}
  \hline
  0 & 1 & 1 & 0\\
  0 & 1 & 0 &  1\\
  0 & 0 & 1 & 1\\
  \hline
	  $c_1$ & $c_2$ & $c_3$ & $c_4$ \\
  \hline
\end{tabular}

$\{c_1, c_2, c_3, c_4\}$ n'est pas indépendant

$c_2 + c_4 = c_3$

$\{c_2, c_4\}$ est indépendant

$\{c_1, c_2, c_4\} $est une base

\subsection{}
Heredité : (axiome 1) :
Supposons que $I' \in I$ non indépendant

$\exists c \in I' $ tel que $ c = \sum_{c_i \in I'}^{} \alpha_i.c_i  $ avec $c_i \in \mathbb{K}$	

cette somme est vrai dans $\mathcal{I}$

I n'est pas indépendant
\end{document}
